\documentclass[prl,aps,reprint,noshowpacs,superscriptaddress,floatfix,letterpaper,longbibliography]{revtex4-2}
\usepackage{amsmath,amssymb,amsbsy,amsfonts,amsthm,bbm,bm,mathtools,mathrsfs}
\usepackage{color}
\usepackage{physics}
\usepackage{xfrac}
\usepackage[dvipsnames]{xcolor}
\definecolor{LapisLazuli}{RGB}{47, 102, 169}
\usepackage[colorlinks=true,citecolor=LapisLazuli,linkcolor=LapisLazuli,urlcolor=LapisLazuli]{hyperref} 

\usepackage{empheq}
\usepackage{pgfplots}
\usepackage{stackengine}
\usepackage{relsize}
\usepackage[inline]{enumitem}
\usepackage[normalem]{ulem} % Comment this out before submitting papers! (otherrwsie will cause problems with references. 
\usepackage{comment}
\usepackage{import}
\usepackage[english]{babel}
\usepackage{lipsum}  

\usepackage{tikz}
\usetikzlibrary{decorations}
\usetikzlibrary{decorations.pathreplacing}

\renewcommand{\bibnumfmt}[1]{(#1)}
\usepackage{enumitem} 

% Macros
\newcommand{\sdc}[1]{\textcolor{blue}{[SD: #1]}}
\newcommand{\jrg}[1]{\textcolor{red}{#1}}
\newcommand{\jrgc}[1]{\textcolor{red}{[JRG: #1]}}
\newcommand{\PlaceHolder}{\blue{[TMP PLACE HOLDER]}} 

% Colors: 
\newcommand{\red}{\textcolor{red}}
\newcommand{\green}{\textcolor{green}}
\newcommand{\blue}{\textcolor{blue}}
\newcommand{\yellow}{\textcolor{yellow}} 
\newcommand{\magenta}{\textcolor{magenta}}

% Shortcuts
\newcommand{\speedtime}{\tau_{\!\scriptscriptstyle\mathcal{A}}} 
\newcommand{\speedtimework}{\tau_{\!\scriptscriptstyle\mathcal{W}}}
\newcommand{\Qdot}{\dot{\mathcal{Q}}}
\newcommand{\Wdot}{\dot{\mathcal{W}}}
\newcommand{\Q}{\mathcal{Q}}
\newcommand{\W}{\mathcal{W}}

% Bold letters
\newcommand{\bp}{\boldsymbol{p}}
\newcommand{\bq}{\boldsymbol{q}}
\newcommand{\bx}{\boldsymbol{x}}
\newcommand{\bC}{\boldsymbol{C}}

% Mathand Tables 
\newcommand{\cov}{\operatorname{cov}} % To mark covariance 
\newcommand{\Hquad}{\hspace{0.5em}} % Short spacing in equations
\newcommand{\HHquad}{\hspace{0.25em}} % Even shorter spacing 
\usepackage{pbox} % For breaking lines in table
\usepackage{tcolorbox}  % For a gray background textbox



\begin{document}
	
	\title{Paper title here}  
	
	\author{Author's name}
	\author{Jason~R.~Green}
	\email[]{jason.green@umb.edu}
	\affiliation{Department of Chemistry,\
		University of Massachusetts Boston,\
		Boston, MA 02125
	}
	\affiliation{Department of Physics,\
		University of Massachusetts Boston,\
		Boston, MA 02125
	}
	\date{\today}
	
\begin{abstract}	
	Abstract goes here.
	\lipsum[1-1]
\end{abstract}

\maketitle

\section{Introduction}
Introduction here Ref~\cite{Nicholson2020}

\lipsum[2-3]

\section{Results and Discussion}
Describe your results.  Here are some useful Latex tips, compare the code with the result in the compiled file. 

\begin{enumerate}
    \item  Look at the spacing and the use of ``align'': 
\begin{align}
\Delta t \bar S/n&\geq k_B.\nonumber\\ 
&=\quad\text{Boltzmann's constant}\qquad >0\HHquad >1 . 
\label{EqSOvern}
\end{align}
Remember to give relevant names to your equations! Not just Eq1, Eq2 etc., because these might change. 
\item Use (backslash) + \textbf{left} and \textbf{right} before parenthesise, to make their size adjustable: 
\begin{align}
[(\frac{\Qdot}{2})],\quad\text{versus}\quad \left[\left(\frac{\Qdot}{2}\right)\right] 
\label{EqParenthesiseSizes}
\end{align}
and here is how you reference to Eq.~\eqref{EqParenthesiseSizes}. 
This also work with floor/ceil and bra/kets: 
\begin{align}
\left\langle \frac{\W}{2}\right\rangle,\quad\left\lfloor e^{-\beta\W}\right\rfloor,\quad\text{and}\quad \left\lceil e^{-\beta\W}\right\rceil 
\label{EqParenthesiseSizes}
\end{align}
\item Use the symbol $\sim$ in order to ``stick'' Eq. to the reference (see the line above, \textbf{in the Latex code}). 
\item Use Eq.~\eqref{}, NOT Eq.~\ref{}. For figures and Tables, use Fig.~\ref{} and Table.~\ref{} without parenthesize. 
\item Use $``$ (on the left) and $"$ (on the right) for quotes, NOT $"$ on both sides. 
\item \blue{This} \yellow{example} \magenta{shows} \green{how} \red{to use colors}. for comments, use \blue{[JohnDoe: Comment here.]}. You can also \textbf{bold}, \sout{strike out} and \textit{Italicize} text. 
\item Figure~\ref{fig:plot-label} shows how to add a normal figure and Fig~\ref{fig:TikzExample} shows how to add a figure with two panels using Tikz.  
\item Say ``Figure'' at the start of a sentence and Fig. in the middle.  
\item Table~\ref{Table1} is an example for how to add tables in Latex.
\item \PlaceHolder  
\item \begin{verbatim}
This is how to add codes/algorithms in Latex.
\end{verbatim} 

\item The following shows how to add bullets: 
\begin{itemize}
  \item bullet 1. 
  \item Bullet 2.
  \item[*] Bullet 3. 
  \item[!] Bullet 4. 
  \label{ListBullets}
\end{itemize}

\item To add a footnote~\footnote{All you need to do is this.}. 

\label{ListEnumarated}
\end{enumerate} 
 

\lipsum[2-3]

 %%%%%%%%% Start normal figure 
 \begin{figure}[h!] %t!=top, h!=here, b!=bottom, tbh=Latex choses best
	\centering
	\hspace*{-0.75cm}\includegraphics[width=0.45\textwidth]{sample-plot.pdf}
	\caption{\footnotesize{Caption goes here, in footnote size.}}
	\label{fig:plot-label}
\end{figure}
 %%%%%%%%% End normal figure
 
 %%%%%%%%% Start Tikzfigure
\begin{figure}[t] %t without the "!" is a reccomendationn for top, which Latex may ignore
\centering
\begin{tikzpicture}
\node (image) at (0,-0.125) {
  \includegraphics[width=0.9\linewidth]{Sample-tex-files/MissingFigure.png}
};
\node (image) at (0,-3.75) {
  \includegraphics[width=0.9\linewidth]{Sample-tex-files/MissingFigure.png}
};
\node[text=black] (a) at (-2.3,1.4){\footnotesize{(a)}};
\node[text=black] (b) at (-2.3,-2.0){\footnotesize{(b)}};
\end{tikzpicture} 
\caption{
{\footnotesize{Figure with two panels, using Tikz.}}}
\label{fig:TikzExample} 
\end{figure} 
 %%%%%%%%% End Tikzfigure 

 %%%%%%%%% Start Table 
%\begin{widetext} 
%\begin{center}
\begin{table}[b] 
\footnotesize
%\centering 
\begin{tabular}{| c | c | c |} % (3 columns) 
\hline\hline 
Category & Col$1$ & Col$2$\\
[0.5ex] %
\hline 
Cat1&$1$&$2$\\
%%%%% 
Cat2&$1$&$2$\\
%%%%% 
Cat3&$1$&$2$\\
[0.5ex]
\hline\hline 
\end{tabular} 
\caption{\footnotesize{Example of a Table.}}
\label{Table1} 
\end{table} 
%\end{center} 
%\end{widetext}
%%%%%%%%% End Table 

\section{Conclusions}

Conclude the paper

\lipsum[2-3]
\\ 


\section{Acknowledgments}
\begin{acknowledgments}
% This text is provided by the funding agency. Jason will provide.
This material is based upon work supported by the National Science Foundation under Grant No. ....

\end{acknowledgments}

\bibliography{references} 

\appendix

\section{App. A}
\label{AppAT} 
We briefly explain the derivation of.... 
\begin{align}
    &(\text{Eqs in the App should be numbered A1,A2,B1 etc.}\nonumber\\ 
    &\text{Need to fix this here.)}
\end{align}



\end{document}
