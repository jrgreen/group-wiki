\documentclass[12pt,a4paper,final]{article}
%\documentclass[prl,aps,preprint,noshowpacs,superscriptaddress,floatfix,letterpaper,longbibliography]{revtex4-2}
\usepackage{amsmath,amssymb,amsbsy,amsfonts,amsthm,bbm,bm,mathtools,mathrsfs}
\usepackage[a4paper, total={6.5in, 9.5in}]{geometry}
\usepackage{color}
\usepackage{physics}
\usepackage{xfrac}
\usepackage[dvipsnames]{xcolor}
\definecolor{LapisLazuli}{RGB}{47, 102, 169}
\usepackage[colorlinks=true,citecolor=LapisLazuli,linkcolor=LapisLazuli,urlcolor=LapisLazuli]{hyperref}
\usepackage{empheq}
\usepackage{pgfplots}
\usepackage{stackengine}
\usepackage{relsize,mathtools}
\usepackage[inline]{enumitem}
\usepackage[normalem]{ulem}
\usepackage{comment}
\usepackage{import}
\usepackage{empheq}
\usepackage{float,graphics}
\usepackage{xr}
\usepackage[english]{babel}
\newcommand{\jacobian}{{\bm{M}}}
\UseRawInputEncoding
% Tikz
\usepackage{pgfplots}
\pgfplotsset{compat = newest}
\usetikzlibrary{arrows,intersections}
\usepackage{tikz-3dplot}
\usepackage{tikz}
\newcommand*{\citen}[1]{%
	\begingroup
	\romannumeral-`\x % remove space at the beginning of \setcitestyle
	\setcitestyle{numbers}%
	\cite{#1}%
	\endgroup   
}
\usepackage[mathscr]{euscript}
\newcommand{\sda}[1]{\textcolor{blue}{[#1]}}
\newcommand{\sdn}[1]{\textcolor{Emerald}{[#1]}}
\newcommand{\jrg}[1]{\textcolor{red}{#1}}
\newcommand{\jrgc}[1]{\textcolor{red}{[#1]}}
\newcommand{\referee}[1]{\textit{#1}}
\newcommand{\bphi}{{\bm{\phi}}}
\newcommand{\bPhi}{{\bm{\Phi}}}
\newcommand{\bpsi}{{\bm{\psi}}}
\newcommand{\bPsi}{{\bm{\Psi}}}
\newcommand\alignboxed[2]{\rlap{\boxed{#1#2}}\hphantom{#1\mkern6mu}}
\usepackage{calligra}
\usepackage{import}
\DeclareMathAlphabet{\mathcalligra}{T1}{calligra}{m}{n}
\DeclareFontShape{T1}{calligra}{m}{n}{<->s*[2.2]callig15}{}
\newcommand{\scriptr}{\mathcalligra{r}\,}
\newcommand{\boldscriptr}{\pmb{\mathcalligra{r}}\,}

\graphicspath{{../plots/}} %Setting the graphicspath

% Tikz
\usepackage{pgfplots}
\pgfplotsset{compat = newest}
\usetikzlibrary{arrows,intersections}
\usepackage{tikz-3dplot}
\usepackage{tikz}


%\input{filename} imports the commands from filename.tex into the target file; it's equivalent to typing all the commands from filename.tex right into the current file where the \input line is.
%
%\include{filename} essentially does a \clearpage before and after \input{filename}, together with some magic to switch to another .aux file, and omits the inclusion at all if you have an \includeonly without the filename in the argument. This is primarily useful when you have a big project on a slow computer; changing one of the include targets won't force you to regenerate the outputs of all the rest.
%
%\include{filename} gets you the speed bonus, but it also can't be nested, can't appear in the preamble, and forces page breaks around the included text.

\begin{document}

\definecolor{umbblue}{RGB}{13,92,145}

\begin{figure}
	\centering
	%\includegraphics[width=\textwidth]{./umb.png}
\end{figure}{}

\vspace{0.25in}

\hfill\today\\

\thispagestyle{empty}

\thispagestyle{empty}

\noindent Dear Dr. Editor,\\

\smallskip

Thank you for considering our paper  for publication.
All three referees seem to have found the work new and ``interesting'' and made suggestions for revising the presentation.
We have put these comments to use in the current version of the manuscript.
%For example, the main result -- a purely classical speed limit for deterministic dynamical systems -- appears on page 2. Page 4 is dedicated to model systems illustrating this result and how it sets the maximum amount of energy that can be dissipated by a system.
We think these improvements address the referees questions and make the paper suitable for the broad audience of your journal.
%We have added both theoretical and numerical results to demonstrate this speed limit for the dissipation of energy.

From the second report of Referee A, our presentation seems to have obscured that
% our speed limits do directly apply to any continuous deterministic systems, including those that are mixing.
%The generality of these results and that they are bounds on energy dissipation are, in part, why we feel strongly that they warrant dissemination to a broader audience.  
Referee B suggested applications to another physical system, and Referee C suggested steps to improve the presentation in the manuscript.

To address their comments, we have made considerable revisions in the current version, including:

(1) 

(2) 

(3) 

(4) 




In our view, these additions better demonstrate the broad applicability of our theory and the revisions, we believe, address the referee's concerns and make the manuscript suitable for your journal.

Please find our point-by-point responses (roman) to the referee's comments (italics) below.



\medskip

\noindent Sincerely,\\

%\noindent \includegraphics[width=1.1in]{./signature.pdf}

\noindent Jason~R.~Green\\
Departments of Chemistry and Physics\\
University of Massachusetts Boston\\
jason.green@umb.edu\\
617-287-6136%\\

\newpage

Below is our detailed response to the first referee. We have reproduced the comments verbatim and give responses to them.\\

\bigskip
\noindent\textbf{Response to Referee A (Ref-A)}
%

\begin{enumerate}[start=1,label={\color{black}(\bfseries Ref-A \arabic*):}]
	\item \referee{The first comment.}
	
Our response - see Ref.\cite{Nicholson2020}
	
	\item \referee{The second comment}
	
	We think 
\end{enumerate}




\newpage
Below is our detailed response to the second referee. We have reproduced the comments verbatim and give responses to them.\\

\bigskip
\noindent\textbf{Response to Referee B (Ref-B)}
%

\begin{enumerate}[start=1,label={\color{black}(\bfseries Ref-B \arabic*):}]
	\item \referee{The first comment.}\label{comment-1}
	
	We don't like you.
	
	\item \referee{The second comment}
	
	See our comment~\ref{comment-1}.
\end{enumerate}
s
\bibliographystyle{ieeetr}
\bibliography{references}

\end{document}

\newpage

Below is our detailed response to the first referee. We have reproduced the comments verbatim and give responses to them.\\

\bigskip
\noindent\textbf{Response to Referee A (Ref-A)}
%

\begin{enumerate}[start=1,label={\color{black}(\bfseries Ref-A \arabic*):}]
	\item \referee{The first comment.}
	
Our response - see Ref.\cite{Nicholson2020}
	
	\item \referee{The second comment}
	
	We think 
\end{enumerate}




\newpage
Below is our detailed response to the second referee. We have reproduced the comments verbatim and give responses to them.\\

\bigskip
\noindent\textbf{Response to Referee B (Ref-B)}
%

\begin{enumerate}[start=1,label={\color{black}(\bfseries Ref-B \arabic*):}]
	\item \referee{The first comment.}\label{comment-1}
	
	We don't like you.
	
	\item \referee{The second comment}
	
	See our comment~\ref{comment-1}.
\end{enumerate}
s

% Bibiliography
\bibliographystyle{ieeetr}
\bibliography{references}

\end{document}
